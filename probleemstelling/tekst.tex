% Created 2014-11-24 Mon 13:15
\documentclass[11pt]{article}
\usepackage[utf8]{inputenc}
\usepackage[T1]{fontenc}
\usepackage{fixltx2e}
\usepackage{graphicx}
\usepackage{longtable}
\usepackage{float}
\usepackage{wrapfig}
\usepackage{rotating}
\usepackage[normalem]{ulem}
\usepackage{amsmath}
\usepackage{textcomp}
\usepackage{marvosym}
\usepackage{wasysym}
\usepackage{amssymb}
\usepackage{hyperref}
\tolerance=1000
\author{Ward Schodts, Xavier Goás Aguililla}
\date{maandag 10 november 2014}
\title{Internet of Things code deployment metrics: probleemstelling}
\hypersetup{
  pdfkeywords={},
  pdfsubject={},
  pdfcreator={Emacs 24.4.1 (Org mode 8.2.10)}}
\begin{document}

\maketitle

\section{Observatie}
\label{sec-1}
In WSNs wordt weinig gebruik gemaakt van de ingebedde rekenkracht en
opslagcapaciteit van de sensor nodes.\\

Mogelijke vragen: 

\begin{itemize}
\item Is dit zo? Waar haal je dit vandaan?\\
  Papers.
\end{itemize}
\section{Vraagstelling}
\label{sec-2}
Zou het mogelijk zijn om een andere aanpak te gebruiken, d.i. meer
berekeningen uit te voeren en meer data op te slaan op de sensor nodes
zelf, zonder verlies van performantie of energie-efficiëntie?\\

Mogelijke vragen: 

\begin{itemize}
\item Geef eens een voorbeeld van zo'n operaties?\\
  Bijvoorbeeld compressie van sensordata en opslag om zo meer data in
één keer op te sturen, maar minder frequent.
\end{itemize}
\section{Waarom}
\label{sec-3}
Als dit mogelijk was dan zouden we minder gebruik moeten maken van de
RF-antennes, die veel energie verbruiken. Eventueel kan dit zelfs ook
betere performantie opleveren, omdat er dan minder data over het
netwerk zou moeten overgedragen worden.
\section{Hypothese}
\label{sec-4}
Wij denken dat in bepaalde scenario's het beter is om bewerkingen op
de sensor nodes uit te voeren voor een efficiënter energiegebruik en
misschien zelfs betere performantie.\\

Mogelijke vragen: 

\begin{itemize}
\item Waarom denk je dit?\\
  Vanuit theoretisch oogpunt lijkt het logisch dat er bewerkingen zijn
die beter wat opgeslagen worden op de nodes berekend. Bijvoorbeeld:
temperatuurmetingen, die we zouden kunnen comprimeren en in grote
'batches' doorsturen.
\end{itemize}
\section{Bewijs}
\label{sec-5}
Metingen die aantonen dat voor een gelijkaardige fysieke setup een
andere verdeling van de werklast binnen het WSN voordelen oplevert qua
energieverbruik en/of performantie vis-à-vis een klassiekere
werkverdeling.\\

Mogelijke vragen: 

\begin{itemize}
\item Is dat zo voor elke andere verdeling?\\
  Zal zich nog uitwijzen tijdens het onderzoek.
\end{itemize}
\section{Vereisten}
\label{sec-6}
Het ontwikkelen van een tool die toelaat om de potentiële
energie-efficiëntie en performantie van een WSN te meten op basis van
welke types code (bv. adhv. vooraf gebenchmarkte API calls) waar
worden uitgevoerd.

Mogelijke vragen: 

\begin{itemize}
\item Hoe willen jullie dit concreet aanpakken?\\
  Zie pyramide.
\end{itemize}
% Emacs 24.4.1 (Org mode 8.2.10)
\end{document}