\documentclass[11pt]{article}
\usepackage[utf8]{inputenc}
\usepackage[T1]{fontenc}
\usepackage{fixltx2e}
\usepackage{graphicx}
\usepackage{longtable}
\usepackage{float}
\usepackage{wrapfig}
\usepackage{rotating}
\usepackage[normalem]{ulem}
\usepackage{amsmath}
\usepackage{textcomp}
\usepackage{marvosym}
\usepackage{wasysym}
\usepackage{amssymb}
\usepackage{hyperref}
\tolerance=1000
\usepackage{todonotes}
\author{Xavier Goás Aguililla}
\date{\today}
\title{Reviews paper WV}
\begin{document}

\maketitle
% \tableofcontents
\section{Review 1 (student)}

\subsection{Samenvatting beoordeling:}
\todo[inline]{grafieken zijn leesbaarder gemaakt ma deze overdrijft wel een beetje}
De paper leest over het algemeen redelijk vlot. De zinnen zijn soms wat lang.
Sommige zinnen moet je ook twee maal lezen voor je weet wat er staat. De
verhaallijn is heel duidelijk. De aanpak, de parameters en de experimenten
worden heel goed toegelicht. De resultaten aflezen uit de grafieken is echter
heel moeilijk. Ik heb namelijk met moeite ontdekt dat er meerdere lijnen op de
grafieken staan. Bovendien ontbreekt ook een legende. De doelstelling van het
werk is gehaald en er wordt duidelijk toegelicht waarom. Bovendien gaan ze een
stap verder door te vermelden hoe ze het nog zouden kunnen verbeteren.

\subsection{Een drietal sterke punten:}

\begin{itemize}
\item Duidelijke verhaallijn
\item Goede titels
\item Zeer goed abstract
\end{itemize}

\subsection{Een drietal zwakke punten:}
\begin{itemize}
\item Figuren redelijk onduidelijk en er ontbreekt een legende - Vaak lange, soms onduidelijke zinnen

\item Te kort, 6.5 bladzijden inclusief referenties
\end{itemize}

\subsection{Inhoudelijke vragen \& suggesties voor verbetering:}

\begin{itemize}
\item Verander de verwijzingen naar de figuren. Het is aangenamer als je ziet staan: "dit resultaat vind je terug op figuur 1" ipv "dit resultaat vind je terug op 1"
\item Op pagina 1 in de rechtse kolom vanboven ontbreekt een $\backslash$"e in effici$\backslash$"ent (waarschijnlijk typfoutje), er zijn ook nog enkele andere typfoutjes.

\item Onder puntje 3.2 worden de parameter schattingen opgesomt. In het begin was mij dit onduidelijk waarvoor de indent diende. Misschien een andere aanpak mogelijk?
\item Probeer de witruimte een beetje te verminderen
\end{itemize}

\section{Review 2 (student)}
\subsection{Samenvatting beoordeling:}
De paper leest vlot en is aangenaam op het vlak van taalgebruik. De lijn van het werk is duidelijk en wordt goed gevolgd in de paper. De doelstellingen van het werk zijn behaald en kunnen hopelijk verder gebruikt worden. De conclusie en redeneringen die gemaakt worden zijn altijd geargumenteerd en verantwoord door de data. Een aantal kleine schrijffouten en bepaalde zinsconstructie die beter kan. De eerste figuur in de paper heeft geen onderschrift en wordt niet genummerd. De paper bevat op zich redelijk weinig tekst.

\subsection{Een drietal sterke punten:}
\begin{itemize}
\item Vlot leesbaar
\item Duidelijke verhaallijn
\item Experimenten zijn duidelijk en worden goed verklaard
\end{itemize}

\subsection{Een drietal zwakke punten:}
\begin{itemize}
\item Redelijk weinig tekst
\item Op bepaalde plaatsen zinsconstructie en spelling
\item Te veel witregels tussen de opsomming van bepaalde factoren.
\end{itemize}

\subsection{Inhoudelijke vragen \& suggesties voor verbetering:}

Toevoegen van een legende bij figuur 3.

De schijnbaar willekeurige 10$^{\text{-3}}$ boven figuur 5.

\section{Review 3 (docent)}

\subsection{Samenvatting beoordeling:}

De paper is goed geschreven en leest vlot. Ook de motivatie en doelstelling is
duidelijk geformuleerd. Hoe de hoofdbijdrage juist werkt is daarentegen vaag
beschreven. Uit de formule is het niet duidelijk hoe alle parameters hierin
worden gebruikt. Iets meer inzicht in hoe de metriek berekend wordt zou nuttig
zijn (o.a. om af te leiden welke parameters het meeste impact hebben).

\subsection{Een drietal sterke punten:}
\begin{itemize}
\item Goed geschreven
\item Goed gemotiveerd
\item Interessante observaties (o.a. impact van opstarten antenne)
\end{itemize}

\subsection{Een drietal zwakke punten:}
\begin{itemize}
\item Voorgestelde metriek is niet geheel duidelijk
\item Grote grafieken met beperkte informatie
\end{itemize}

\subsection{Inhoudelijke vragen \& suggesties voor verbetering:}

\begin{itemize}
\item Een paar typefouten: ‘efficint’
\item Onnodig gebruik van voetnoten: de tweede kan vervangen worden door de citatie in de tekst te plaatsen en de derde kan ook in de tekst en voorkomt dat de voetnaat als ‘tot de derde macht’ wordt gelezen.
\item abstract: Het woord ‘mote’ is nog niet geïntroduceerd en terminologie.
\item 1.1: ‘concurrent’ is geen Nederlands in deze context
\item 2, formule: ‘waarbij n …’ Er komt geen n of m voor in de formule. ‘tr()’ en ‘red()’ daarentegen worden niet formeel geïntroduceerd. Hoe wordt ‘cost()’ berekend? Hoe zorg je er voor dat tr() en cost() dezelfde grootte-orde uitdrukken?
\item 3.2, laatste zin: Wat zijn die edge cases?
\item 3.4: ‘Paremeters.’ is verwarrende formattering want ziet er hetzelfde uit als de paragraaf hoofdingen.
\item Fig 3: legende uitbreekt
\item Fig 5: Waarom geen grafiek\todo[inline]{done} met antenne-verbruik per byte? Lijkt interessant aangezien je meldt dat dit niet-linear is. Je zegt daarentegen in ‘verzenden en ontvangen’ wel dat het verzenden ‘meer’ kost maar nog ‘redelijk’. Dit soort vage termen zijn te mijden aangezien ze niets vertellen, gebruik concrete getallen.
\item 5: Je gebruikt voor de eerste keer hier de terminologie ‘20B’.
\end{itemize}

\section{Review 4 (Klaas Thoelen)}

\subsection{Samenvatting beoordeling (een 5 a 10 lijntjes):}
De structuur van de paper is grotendeels duidelijk en laat vlot lezen toe. Op
bepaalde punten blijft de paper echter vaag over specifieke stappen in het
onderzoek en de resultaten ervan (zie hieronder). Hier zou wat meer uitleg de
paper nog verder verduidelijken. Het besproken werk levert een substantiële, en
voldoende, contributie aan de beoogde problematiek en tevens degelijke ideeën
voor verder werk. Wat ontbreekt is een evaluatie van de voorgestelde tool en
bijhorende metrieken. Hoe accuraat is de aanbeveling die door de tool wordt
gegenereerd? Hoeveel overhead is er voor de ontwikkelaar om de parameters voor
de tool te bepalen? In het algemeen geeft de paper blijk van weldoordacht
onderzoek en bijhorende oplossingen en aanbevelingen.



\subsection{Een drietal sterke punten:}
\begin{itemize}
\item goede afhandeling (argumentatie) in de tekst van mogelijke variabiliteit in te bepalen parameters.
\item grondige methodologie in het onderzoek
\end{itemize}

\subsection{Een drietal zwakke punten:}
\begin{itemize}
\item de tool zelf geeft niet dadelijk een oplossing voor variabiliteit van invoerparameters
\item soms wat vaag, nieuwe elementen in de tekst worden niet of te weinig uitgelegd
\item het lijkt erop dat het steeds beter is om berekeningen te doen op de mote; er worden weinig andere situaties aangehaald, wat de contributie van het onderzoek minimaliseert
\item geen evaluatie van de uiteindelijke contributie. Hoe accuraat is de voorgestelde tool?
\end{itemize}

\subsection{Inhoudelijke vragen \& suggesties voor verbetering:}

Weinig inhoudelijke vragen. Maar de tekst zelf zou nog wat duidelijker kunnen voor de lezer die niet vertrouwd is met het werk. Sequentieel doorheen de paper:
\begin{itemize}
\item minder wij-vorm gebruiken; eerder ‘Deze paper beschrijft …’
\item ‘zonder lokaal energie- of performantieverlies’
\item Tussen sectie 1 en 1.1: geen 2 titels dadelijk na elkaar; plaats er iets tekst tussen; bv. wat er in de volgende 2 subsecties zal worden besproken
\item idem sectie 3.2
\item Is het mogelijk om een simpele, algemene maatstaf te gebruiken? Beantwoord deze vraag uit de inleiding expliciet in de conclusie
\item voetnoot 2: vermijd ‘leuk’, haal dit scenario kort aan in tekst (kan in 1 zin) of geef eerder al een dergelijk voorbeeld van zo een WSN. Liever concreet en kort, dan een vage verwijzing.
\item spelling ‘efficiënt’
\item het is niet echt duidelijk over welke applicaties jullie spreken: probeer in de inleiding concrete voorbeeld(en) te geven: gemiddelde, aggregatie, …
\item check op spelling, formatting, …!!
\item intro Sectie 2 is onduidelijk; geef eerst aan dat een vuistregel wordt uitgewerkt en dat deze wordt geautomatiseerd in een tool; bouw dan pas de vuistregel op
\item er staat geen n en m in vuistregel; wat doet de functie ‘compute’?; wat is tr, cost, red, x?
\item plaats de tekstuele uitleg van de vuistregel vóór de wiskundige uitwerking
\item eerste parameter is ‘sensordata’, niet ‘sensoren’
\item specifieer dat geheugen over RAM of Flash gaat
\item ‘Tijdsinformatie’ = ‘frequentie’?
\item ‘kostfunctie’ = ‘rekenkost’?
\item ‘reductiegrootte’ = ‘reductiefactor’?
\item begin sectie 3: vermijd historisch sequentieel verhaal
\item storage, computation, transmission: nederlands
\item een weerstand wordt niet aan/uit gezet
\item wat is het pinout signaal?
\item vermeld hoe je de pin-flips hebt gerealiseerd (eventueel met code)
\item figuur met output van oscilloscoop (meetresultaat en pinflips)?
\item waarom geeft het fysiek meten een harde bovengrens voor de CPU-cycli?
\item Geheugenverbruik, sensoren, precisie: welke edge cases?
\item wat is een deployer?
\item waarom eerst programma schrijven en dan pas parameterbestand aanmaken? Kan men niet voor het programmeren al testen of de geplande aanpak efficient is of niet?
\item ‘één temperatuurmeting per minuut’
\item zie dat alle parameters en hun waarden duidelijk zijn voor de lezer
\item vanwaar 1639mJ??
\item ‘Deployen op mote.’ is geen zin. Geef ook wat meer context. Waarom betekent dit deployen op mote?
\item Sectie resultaten evalueert jullie oplossing niet, maar is de input voor jullie oplossing. Plaats deze tussen 3.1 en 3.2.
\item verwijs naar figuren mbv ‘figuur 1’, niet enkel ‘1’
\item zorg dat de figuren duidelijk zijn en correct geïnterpreteerd worden; geef wat meer uitleg over assen, meetresultaten, etc.
\item 1ms of 1s voor 3000 bytes? (zie Y-as figuur 2)
\item figuren zijn allemaal rechte lineaire grafieken: zit er nergens variatie op? Lijkt wat artificieel. \todo[inline]{is gewoon de realiteit}
\item figuur 3 en 4: meerdere datasets; voeg een legende toe die aangeeft wat iedere grafiek weergeeft.
\item conclusie: vermijd ‘we’, herhaal kort de eisen

\item gedurende de hele tekst worden stap-per-stap nieuwe problemen, vragen, etc. geïntroduceerd; probeer dit te vermijden en de lezer van in het begin duidelijk te maken waarover hij later zal lezen. Bv. over effectieve metingen om de vuistregel op te staven wordt pas in sectie 3.1 gesproken.
\end{itemize}
\end{document}