\documentclass{article}
\usepackage{ijcai11}
\usepackage{lipsum}
\usepackage{times}
\usepackage{latexsym} 
\usepackage{todonotes} 
\usepackage[dutch]{babel} 

\title{Internet of Things code deployment metrics}
\author{Ward Schodts \and Xavier Go\'as Aguililla \\ ward.schodts@student.kuleuven.be \\ xavier.goas@student.kuleuven.be}

\begin{document}

\maketitle

\begin{abstract}
Wij stellen een simpele vuistregel voor die kan dienen om te beslissen of een
bepaald stuk applicatielogica op een mote kan draaien zonder energie- of
performantieverlies.
  
% \lipsum[1]
\end{abstract}

\section{Inleiding \& probleemstelling}

\todo[inline]{explain what WSNs are?}

Energie-effici\"entie is een cruciale factor op alle niveau's bij het
ontwikkelen van wireless sensor networks: een typische mote heeft geen toegang
tot een onbeperkte stroombron en moet het doen met een batterij. Deze batterij
kan in veel scenario's waarin WSNs worden gebruikt ook niet hernieuwd worden, en
dus is de levensduur van de mote ook afhankelijk van hoe zuinig hij omspringt
met energie. Het is dan ook geen wonder dat veel research in het gebied
rechtstreeks wordt be\"invloed door deze kwestie: van de ontwikkeling van
besturingssystemen voor motes over netwerkprotocollen tot studies van
netwerktopologie\"en.

Een typische mote heeft ook erg beperkte reken- en opslagcapaciteit, wat
verhindert dat er meerdere processen op effici\"ente wijze concurrent kunnen
worden uitgevoerd. In het algemeen gaat men daarom een simpele strategie
toepassen voor dataverwerking, waarbij de mote \'e\'en enkele
verantwoordelijkheid heeft, het zgn. 'sense and send': sensordata wordt op de
motes niet bewerkt, maar meteen doorgestuurd naar de backend voor verdere
verwerking, waardoor de rol van de motes bij het verwerken van de data
geminimaliseerd wordt.

Dit is de na\"iefste aanpak die men kan gebruiken, en steunt net op het deel van
mote dat het gulzigst is met energie: de antenne.  De vraag dringt zich op: is
er geen manier om van de weliswaar beperkte rekenkracht van de motes gebruik te
maken om effici\"enter om te springen met de antenne? En is het mogelijk een
simpele, algemene maatstaf te gebruiken om te beslissen of dit in een specifiek
deployment scenario kan of niet?

\section{Methodologie}

% \lipsum[3]

\section{Resultaten}
% \lipsum[4]
\section{Conclusie \& verder werk}

% \lipsum[5]

%% The file named.bst is a bibliography style file for BibTeX 0.99c
\bibliographystyle{named}
\bibliography{bibliography}

\end{document}

