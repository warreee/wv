\documentclass{article}
\usepackage{ijcai11}
\usepackage{lipsum}
\usepackage{times}
\usepackage{latexsym} 
\usepackage{todonotes} 

\title{Internet of Things code deployment metrics}
\author{Ward Schodts \and Xavier Go\'as Aguililla \\ ward.schodts@student.kuleuven.be \\ xavier.goas@student.kuleuven.be}

\begin{document}

\maketitle

\begin{abstract}
We propose a simple, rule-of-thumb metric for deciding where to deploy
functionality in wireless sensor networks. 
  
\lipsum[1]
\end{abstract}

\section{Introduction}

\todo[inline]{explain what WSNs are?}

Energy efficiency is a key concern in the development of wireless
sensor networks. Typically, WSN motes do not have access to an
unlimited power source, but have to make do with batteries. It is
desirable for the motes to make good use of these batteries in order
for the network to have a long lifespan. This has made research on
energy efficiency an important part of research on WSNs in general,
impacting OS development, network protocols, topology studies ...

WSN motes themselves have very little computational power and storage space, and
are bad at concurrency. Therefore, the most widely used type of code deployed to
WSN motes is `sense and send', in which motes do one thing: transmit the data
coming from their sensors as it comes in. This, of course, is the most naive
method possible.

\section{Materials and methods}

\lipsum[3]

\section{Results}
\lipsum[4]
\section{Conclusion \& future work}

\lipsum[5]

%% The file named.bst is a bibliography style file for BibTeX 0.99c
\bibliographystyle{named}
\bibliography{bibliography}

\end{document}

