\documentclass{article}
\usepackage{ijcai11}
\usepackage{lipsum}
\usepackage{times}
\usepackage{latexsym} 
\usepackage{todonotes} 
\usepackage{amsmath} 
\usepackage[dutch]{babel} 

\title{Internet of Things code deployment metrics}
\author{Ward Schodts \and Xavier Go\'as Aguililla \\ ward.schodts@student.kuleuven.be \\ xavier.goas@student.kuleuven.be}

\begin{document}

\maketitle

\begin{abstract}
Wij stellen een simpele vuistregel voor die kan dienen om te beslissen of een
bepaald stuk applicatielogica op een mote kan draaien zonder lokaal energie- of
performantieverlies.
  
\end{abstract}

\section{Situering \& probleemstelling}

\todo[inline]{explain what WSNs are?}

\subsection{Wat}

Energie-effici\"entie is een cruciale factor op alle niveau's bij het
ontwikkelen van wireless sensor networks: een typische mote heeft geen toegang
tot een onbeperkte stroombron en moet het doen met een batterij. Deze batterij
kan in veel scenario's waarin WSNs worden gebruikt ook niet hernieuwd worden, en
dus is de levensduur van de mote ook afhankelijk van hoe zuinig hij omspringt
met energie. Het is dan ook geen wonder dat veel research in het gebied
rechtstreeks wordt be\"invloed door deze kwestie: van de ontwikkeling van
besturingssystemen voor motes over netwerkprotocollen tot studies van
netwerktopologie\"en.

Een typische mote heeft ook erg beperkte reken- en opslagcapaciteit, wat
verhindert dat er meerdere processen op effici\"ente wijze concurrent kunnen
worden uitgevoerd. In het algemeen gaat men daarom een simpele strategie
toepassen voor dataverwerking, waarbij de mote \'e\'en enkele
verantwoordelijkheid heeft, het zgn. 'sense and send': sensordata wordt op de
motes niet bewerkt, maar meteen doorgestuurd naar de backend voor verdere
verwerking, waardoor de rol van de motes bij het verwerken van de data
geminimaliseerd wordt.

Dit is de na\"iefste aanpak die men kan gebruiken, en steunt net op het deel van
mote dat het gulzigst is met energie: de antenne.  De vraag dringt zich op: is
er geen manier om van de weliswaar beperkte rekenkracht van de motes gebruik te
maken om effici\"enter om te springen met de antenne? En is het mogelijk een
simpele, algemene maatstaf te gebruiken om te beslissen of dit in een specifiek
deployment scenario kan of niet?

\subsection{Waarom?}


\subsection{Succescriteria}

Een eenvoudige vuistregel, die:
\begin{itemize}
\item In de grote meerderheid van de gevallen een juiste beslissing zal maken;
\item eenvoudig te integreren is in de ontwikkelingscyclus;
\item semi-automatisch werkt.
\end{itemize}

\section{Voorgestelde oplossing}
We vertrekken vanuit de veronderstelling dat de ontwikkelaar een stuk
applicatielogica rechtstreeks op de motes wilt deployen. Vanuit deze optiek
maken we abstractie van de low-level details van simulatie e.d., en gebruiken we
in plaats daarvan een eenvoudige vuistregel (\textit{code deployment
metric}. Deze geeft een makkelijke ja/nee-beslissing die zegt of we kunnen
deployen of niet. We kunnen deze uitdrukken als volgt:

\begin{equation}
  \begin{split}
    cost(reduce(input)) + cost(transfer(m)) \\ 
    \leq \\ 
    cost(transfer(n))
  \end{split}
\end{equation}

Waarbij $n$ de grootte van de sensordata in bytes is, en $m$ de verwachte
grootte van de output van de functie compute toegepast op de sensorinput, ook in
bytes.

Wat deze ongelijkheid afweegt is: gegeven een reductie van de sensordata van $n$ naar
$m$ bytes, is de som van de kost van deze reductie op een input van een bepaalde
grootte en de kost van het verzenden van de output van deze reductie kleiner dan
de kost van het verzenden van de originele input?

De aard van deze reductie ligt niet vast. In eerste instantie zouden we kunnen
denken aan een louter computationeel proces dat het aantal over te dragen bytes
reduceert en dan meteen overdraagt. Zo'n voorbeeld is een filter: we sturen een
meting pas door als die verschilt van de vorige verstuurde meting. Maar we
kunnen evengoed de sensordata aggregeren in het geheugen, zodat de transferkost
wanneer een meting binnenkomt nul wordt, behalve als de buffer waarin we data
opslaan vol is.\footnote{Merk ook op dat de kost van het verzenden van gegevens
niet lineair is -- er is geen garantie dat $cost(transfer(x)) +
cost(transfer(y)) = cost(transfer(x + y))$. In een keer $x + y$ bytes versturen
kan dus voordeliger uitkomen dan $x$ en $y$ bytes apart te verzenden.} 
Ook een combinatie van beide is mogelijk: we kunnen bijvoorbeeld de buffer
comprimeren door reeksen identieke metingen samen te nemen (filter +
aggregator). Dit zou theoretisch gezien een grotere winst moeten opleveren.

De aandachtige lezer stelt zich meteen de vraag: hoe kunnen wij de kost van zo'n
reductie in het algemeen schatten, zuiver uitgaand van het programma? 

\todo{uitweiden}

We laten de programmeur, die zijn programma goed kent, ingrijpen. We nemen als
input niet enkel het gecompileerde programma, maar ook een set parameters. Deze zijn bijvoorbeeld:

\paragraph{Sensoren.} Hoe groot is de sensordata? Hoe vaak komt er sensordata binnen?
\paragraph{Geheugenverbruik.} Worden er gegevens bijgehouden in een buffer? Hoeveel? Hoe vaak wordt er naar die buffer geschreven?
\paragraph{Nauwkeurigheid.} Wat is de gewenste granulariteit van de metingen,
d.w.z. hoe groot mag het interval tussen twee doorgestuurde metingen zijn?

\section{Methodologie \& materiaal}

De kwestie is nu om wat we theoretisch uiteen hebben gezet in te vullen met
empirische data: hoe berekenen wij de kost van het verzenden en verwerken van
gegevens concreet? We berekenen het kostenplaatje \textit{atomisch}: we kiezen
bepaalde eenheden voor de drie componenten van het energieverbruik die we kunnen
be\"invloeden:

\begin{itemize}
\item \textbf{storage}: hoeveel kost het om $n$ bytes op te slaan in het RAM of
het EEPROM?
\item \textbf{computation}: hoeveel kosten $n$ seconden CPU-tijd?
\item \textbf{transmission}: hoeveel kost het om $n$ bytes te verzenden; hoeveel
kost het om de antenne aan en uit te zetten?
\end{itemize}

Een eerste reflex is om naar de technische fiche van de motes te grijpen. die
zijn dan wel vrij betrouwbaar, maar geven metingen weer die als het ware in een
vacu\"um zijn gedaan: de kost van het draaien van het OS, van het op laten staan
van de antenne in low power listening modus, enz. worden hierin niet
meegerekend. We zijn daarom experimenteel gaan uitzoeken wat de kost was van elk
van deze types.

\subsection{Meetopstelling en materialen}

Voor onze metingen maakten wij gebruik van de AVR Zigduino. Deze werd
aangesloten op een circuit met spanning 6V en stroom 0.04A.

Voor het meten van het energieverbruik gedurende een bepaalde periode maken we gebruik van het volgende circuit:

\begin{figure}[h]
\centering
\missingfigure{}
\caption{De meetopstelling}
\label{fig:meetopstelling}
\end{figure}

De oscilloscoop meet de spanning in het circuit. De weerstand kan aan en uit
worden gezet door de mote door het pin-outsignaal te flippen. Er is enkel stroom
in het circuit als het signaal op hoog staat; die periodes zijn duidelijk af te
lezen op de output van oscilloscoop. We kunnen gebruik maken van de wet van Ohm
om de verbruikte stroom in die intervallen te meten. Daardoor kunnen we met
goede precisie meten hoeveel energie een bepaald stuk code verbruikt.

\subsection{De tool}

\subsection{Een concreet stappenplan voor deployers}

\subsection{Een voorbeeldscenario: temperatuurmetingen}

\section{Resultaten}

\begin{figure}[h]
\centering
\missingfigure{}
\caption{Energieverbruik schrijven naar RAM}
\label{fig:energieverbruik_ram}
\end{figure}

\begin{figure}[h]
\centering
\missingfigure{}
\caption{Tijdsduur schrijven naar RAM}
\label{fig:energieverbruik_ram}
\end{figure}

\begin{figure}[h]
\centering
\missingfigure{}
\caption{Energieverbruik in low-power listening}
\label{fig:energieverbruik_low_power}
\end{figure}

\begin{figure}[h]
\centering
\missingfigure{}
\caption{Energieverbruik met antenne volledig uit}
\label{fig:energieverbruik_antenne_uit}
\end{figure}

\subsection{Kostmetingen}

\subsubsection{Schrijven naar RAM}

\subsubsection{Low-power listening}

\subsubsection{Verzenden en ontvangen}

\subsubsection{Basisverbruik}

\subsubsection{CPU-cycli}

\subsection{Voorbeeldscenario: werkt de metric?}

\section{Conclusie \& verder werk}

\subsection{Volledige netwerktopologie\"en}

Conclusies uitbreiden naar volledige netwereken (past mooi binnen
pyramidestructuur) gebruik maken van aantal hops (indien bvb de node z’n
geheugen vol is, of door het beschouwen van de kost van de transmissie over
meerdere nodes). Bijvoorbeeld: metric zegt nee voor een enkele node, maar de
kost van het reduceren van de output zorgt ervoor dat de transmissiekost over
het netwerk als geheel in functie van het aantal hops groter is dan de kost van
het reduceren van de output. Maak gebruik van zowel de cost van verzenden als
ontvangen.

\bibliographystyle{named}
\bibliography{bibliography}

\end{document}
