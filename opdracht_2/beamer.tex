% Created 2014-11-08 Sat 11:30
\documentclass[presentation, bigger]{beamer}
\usepackage[utf8]{inputenc}
\usepackage[T1]{fontenc}
\usepackage{fixltx2e}
\usepackage{graphicx}
\usepackage{longtable}
\usepackage{float}
\usepackage{wrapfig}
\usepackage[normalem]{ulem}
\usepackage{textcomp}
\usepackage{marvosym}
\usepackage{wasysym}
\usepackage{latexsym}
\usepackage{amssymb}
\usepackage{amstext}
\usepackage{hyperref}
\usepackage{url}
\usepackage[dutch]{babel}
\usepackage[font=scriptsize,labelfont=bf]{caption}
\setbeamertemplate{caption}[numbered]


\tolerance=1000
\usetheme{kuleuven}
\useinnertheme{rectangles}
\graphicspath{{graphics/}}
\usepackage[style=authoryear,hyperref,backref,square,natbib,ibidtracker=false]{biblatex}
\bibliography{bibliography}

\usepackage{graphicx}
\usetheme{default}
\author{Ward Schodts, Xavier Goás Aguililla}
\date{maandag 10 november 2014}
\title{Internet of Things code deployment metrics}
\hypersetup{
  pdfkeywords={},
  pdfsubject={},
  pdfcreator={Emacs 24.3.1 (Org mode N/A)}}



\newcommand{\aheader}[2]{\action<#1-|alert@#1>{#2}}
% first argument: slide number to appear from, second argument: content of header 
\newcommand{\hiddencell}[2]{\action<#1->{#2}}
% first argument: slide number to appear from, second argument: content of cell

\begin{document}

\maketitle
\begin{frame}[noframenumbering]{Outline}
\tableofcontents
\end{frame}

%\begin{itemize}
%\item TODO hier een afbeelding zoeken en aan de hand hiervan uitleggen!
%\item bestaan uit embedded computers, zgn. ‘motes’
%TODO foto/video van motes
%\item uitgerust met low-power radioantennes en sensoren
%\end{itemize}


\section{Introductie}
\label{sec-1}
\begin{frame}[label=sec-1-1]{Wireless sensor networks: wat zijn ze? (1)}

\includegraphics[width=0.9\textwidth,keepaspectration=true]{intro/overview.png}

\end{frame}

\begin{frame}[label=sec-1-2]{Wireless sensor networks: wat zijn ze? (2)}

\includegraphics[width=\textwidth,keepaspectration=true]{intro/psumote.jpg}

\end{frame}



\begin{frame}[label=sec-1-3]{Toepassingen van WSN}
\centering 
\begin{tabular}{c c c}
\includegraphics[width=5.25cm,keepaspectration=true]{graphics/sample_applications/fire.jpg}
& \includegraphics[width=5cm,keepaspectration=true]{graphics/sample_applications/landbouw.jpg} \\
\includegraphics[width=5.25cm,keepaspectration=true]{graphics/sample_applications/military.jpg}
& \includegraphics[width=5cm,keepaspectration=true]{graphics/sample_applications/medicine.jpg}
\end{tabular}
\note{and another test}

\end{frame}

\begin{frame}[label=sec-1-4]{Great Duck Island experiment I}
\centering
\begin{figure}
\fbox{\includegraphics[width=0.85\textwidth,keepaspectratio=true]{gdi/gdi.jpg}}
\caption{Great Duck Island}
\end{figure}

\end{frame}
%TODO Beter woord voor monitoring vinden
\begin{frame}[label=sec-1-5]{Great Duck Island experiment II}
\begin{figure}
\begin{minipage}{.5\textwidth}
\centering
\fbox{\includegraphics[width=.8\textwidth]{gdi/gdi_duckmote.jpg}} 
\captionof{figure}{A figure}
\end{minipage}%
\begin{minipage}{.5\textwidth}
\centering
\fbox{\includegraphics[width=.8\textwidth]{gdi/micamote.png}}
\captionof{figure}{Another figure}
\end{minipage}
\end{figure}
\begin{itemize}
\item habitat monitoring van eenden
\item motes in broedholen
\item 7 maanden
\end{itemize}
\note{Mensen konden niet in de buurt komen zonder te storen}
\note{In de holen werden de motes gelegd}
\note{En aanvullingen werden bijgezet}
\end{frame}

\begin{frame}[label=sec-1-6]{Belangrijke aspecten bij WSN design}
\begin{center}

\vfill
\begin{tabular}{l c r}
\hiddencell{1}{\includegraphics[width=0.25\textwidth,keepaspectration=true]{wsndesign/battery.png}} & \hiddencell{2}{\includegraphics[width=0.25\textwidth,keepaspectration=true]{wsndesign/density.png}} & \hiddencell{3}{\includegraphics[width=0.15\textwidth,keepaspectration=true]{wsndesign/euro.png}}  \\
\hiddencell{1}{Energieverbruik} & \hiddencell{2}{Dichtheid} & \hiddencell{3}{Goedkoop}\\
%
\hiddencell{4}{} & \hiddencell{5}{}\\
\hiddencell{4}{} & \hiddencell{5}{}\\
\hiddencell{4}{Autonoom} & \hiddencell{5}{Adaptief}
\end{tabular}
\vfill

\end{center}
\note{Consume extremely low power, operate in high volumetric densities, have low production cost and be dispensable, be autonomous and operate unattended, be adaptive to the environment.}
\end{frame}

\begin{frame}[label=sec-1-7]{Store, compute, transmit? (1)}
\begin{itemize}
\item drie grote factoren in energieverbruik:
\vfill
\begin{tabular}{c c c}
\hiddencell{2}{\includegraphics[width=0.25\textwidth,keepaspectration=true]{storage}} & \hiddencell{3}{\includegraphics[width=0.25\textwidth,keepaspectration=true]{cpu}} & \hiddencell{4}{\includegraphics[width=0.25\textwidth,keepaspectration=true]{radio}}  \\
\hiddencell{2}{flash-opslag} & \hiddencell{3}{berekeningen} & \hiddencell{4}{netwerkoverdracht}
\end{tabular}
\vfill
\end{itemize}
\end{frame}

\begin{frame}[label=sec-1-8]{Store, compute, transmit? (2)}
\begin{itemize}
\item Mss een grafiekje dat de verschillen duidt?
\end{itemize}
\end{frame}
\section{Middleware voor WSNs}
\label{sec-2}
\begin{frame}[label=sec-2-1]{Wat is middleware?}
\includegraphics[width=\textwidth,keepaspectration=true]{middleware}
\end{frame}

\begin{frame}[label=sec-2-2]{Mogelijke aanpakken}
\begin{itemize}
\item application-based; bv. Contiki, Squawk
\item component-based; bv. OpenCOM, Figaro, LooCi
\begin{itemize}
\item statisch
\item dynamisch reconfigureerbaar
\end{itemize}
\end{itemize}
\end{frame}

\begin{frame}[label=sec-2-3]{LooCi (1)}
\begin{columns}[t]
\column{.5\textwidth}
\centering
\includegraphics[width=5cm,keepaspectration=true]{looci/looci.png}\\
\includegraphics[width=5cm,keepaspectration=true]{looci/distrinet.png}
\column{.5\textwidth}
\centering
\begin{itemize}
\item Ontwikkeld aan KULeuven
\item Runtime deployable components
\item Werkt op Contiki, Sun SPOT, OSGi en Android
\end{itemize}
\end{columns}
\end{frame}

\begin{frame}[label=sec-2-4]{Looci (2)}
\centering
\includegraphics[width=0.9\textwidth,keepaspectration=true]{looci/LooCIExecEnvironment.png}
\end{frame}

\begin{frame}[label=sec-2-5]{Looci Demo}
Filmpje van youtubekanaal
\end{frame}

\section{Energieverbruik analyseren}
\label{sec-3}

\begin{frame}[label=sec-3-1]{Waarom is energiegebruik belangrijk?}
\begin{itemize}
\item rechtstreeks verband met belangrijke aspecten van WSN-design:
\end{itemize}
\end{frame}

\begin{frame}[label=sec-3-2]{Hoe meten we het?}
\begin{minipage}{\textwidth}
\centering
\fbox{\includegraphics[width=0.9\textwidth,keepaspectration=true]{elek/dso.jpg}} 
\captionof{figure}{Oscilloscoop}
\end{minipage}
\end{frame}

\begin{frame}[label=sec-3-3]{Setup}
\includegraphics[width=\textwidth,keepaspectration=true]{elek/diag1}
\end{frame}

\begin{frame}[label=sec-3-4]{Voltageplot}
\includegraphics[width=0.95\textwidth,keepaspectration=true]{elek/energy_measurement_plot.png}
\end{frame}

\begin{frame}[label=sec-3-5]{Analyse energieverbruik}
\includegraphics[width=\textwidth,keepaspectration=true]{elek/diag2}
\begin{itemize}
\item kan afgeleid worden met de wet van Ohm
\item kan gemodeleerd worden m.b.v. lineaire regressie
\end{itemize}
\end{frame}
\section{Conclusie}
\label{sec-4}
\begin{frame}[label=sec-4-1]{Waar komen wij in het spel?}
\begin{itemize}
\item huidige aanpak in het veld: netwerk-overdracht
\item is dit wel zo?
\item implementeren tool voor simulatie energieverbruik
\end{itemize}
\end{frame}

\begin{frame}[label=sec-4-2]{Conclusie}
\end{frame}
\begin{frame}[allowframebreaks]{Bibliografie}
\nocite{*}
\printbibliography
\end{frame}

% Emacs 24.3.1 (Org mode N/A)
\end{document}
